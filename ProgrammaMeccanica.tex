MECCANICA



Grandezze fisiche e loro misura

%Grandezze fisiche e loro definizione operativa. 
%Sistemi di unità di misura e dimensioni; equazioni ed analisi dimensionali. 
Misure di tempo: generalità sul concetto di tempo; “periodicità” e confronto tra fenomeni periodici; orologi e loro sincronizzazione; continuo temporale. 
Misure di lunghezza: regoli campione. 
%Coordinate spaziali.



%Geometria dei vettori

%Spostamenti e vettori: definizione di spostamento. 
%Sistemi di coordinate. 
%Operazioni nell’insieme degli spostamenti: somma, differenza, moltiplicazione per uno scalare, prodotto scalare. 
%Geometria dello spazio fisico: piani e rette fisiche e loro confronto con il modello matematico (spazio euclideo). 
%Definizione di vettore. 
%Operazioni sui vettori: prodotto vettoriale tra vettori. 
%Definizione geometrica delle operazioni e loro rappresentazione con le coordinate.



Cinematica del corpo puntiforme

%Definizione di punto materiale e significato dell’approssimazione. 
%Corpi in traslazione uniforme: in 1 dimensione e poi in 2 o 3 dimensioni.
%Legge oraria e traiettoria di un punto materiale in movimento. 
%Coordinate cartesiane e ascissa curvilinea, grandezze vettoriali e grandezze scalari: definizione di velocità media ed istantanea. 
%Moti vari: definizione di accelerazione media ed istantanea. 
%Moti su una retta, moti piani, rotazioni e loro descrizione, moto armonico. 
%Dall’accelerazione (assegnata in funzione del tempo) alla legge oraria. 
Relazioni tra velocità e accelerazioni in sistemi in moto traslatorio rettilineo accelerato e in rotazione uniforme l'uno rispetto all’altro.



Dinamica del corpo puntiforme

%I principi 
Osservazione di moti di carrelli a basso attrito su piani inclinati. 
%Esperienze di Galileo. 
%Principio di inerzia. 
Osservazione di interazioni tra carrelli. 
Definizione dinamica di massa. 
%Definizione (statica) di forza. 
%Forza e accelerazione. 
%Secondo e Terzo principio della dinamica. 
%Le leggi di forza. 
%Forza elastica ed equazione differenziale: oscillatore armonico. 
%Forza di gravitazione universale. 
Massa inerziale e massa gravitazionale. 
%Sistemi di riferimento inerziali e principio d'inerzia. 
%Principio di relatività: trasformazioni galileiane e invarianza dei principi della dinamica. 
%Sistemi non inerziali e forze apparenti.



Leggi delle forze

%Esempi di forze. 
%La forza peso. 
%Moto dei gravi. 
Forza viscosa. 
%Moto armonico semplice e smorzato. 
%Forze di contatto: reazioni vincolari e forze di attrito. 
%Tensioni nei fili e carrucole. 
%Gravitazione universale: il campo gravitazionale; le leggi di Keplero.
La gravitazione terrestre e gli effetti della rotazione della terra: il peso dei corpi (e inoltre: fili a piombo, pendolo di Foucault, spostamento verso Est).



Conseguenze dei principi della dinamica

%Impulso e quantità di moto. 
%Teorema dell'impulso. 
%Momento angolare e momento della forza. 
%Equazione del momento angolare e conservazione del momento angolare. 
%Lavoro e potenza di una forza. 
%Energia cinetica e Teorema delle forze vive (o dell’energia cinetica).
%Calcolo del lavoro e integrale di linea. 
%Campi di forze conservativi. 
%L'energia potenziale e la conservazione dell’energia meccanica. 
%Campi di forze centrali. 
Moto di missili balistici e satelliti in campo di forza gravitazionale. 
Condizioni di equilibrio per un punto materiale.



Dinamica dei sistemi

%Conseguenze generali dei tre principi della dinamica per i sistemi di molte particelle. 
%Centro di massa, per 2 e per n corpi. 
%Quantità di moto totale e moto del centro di massa. 
%Prima e Seconda equazione cardinale del moto dei sistemi; conservazione della quantità di moto totale e del momento angolare totale. 
%Energia cinetica totale e teorema di König. 
%Generalizzazione del teorema delle forze vive; lavoro delle forze interne. 
%Sistema a due corpi. 
%Sistemi continui.



Meccanica dei corpi rigidi

%Definizioni e cinematica dei corpi rigidi (traslazioni, rotazioni rispetto ad assi fissi, rotolamento, asse istantaneo di rotazione).
%Equazioni cardinali della dinamica dei corpi rigidi. 
%Statica dei corpi rigidi. 
%Energia cinetica di un corpo rigido e momento d'inerzia rispetto ad un asse. 
%Momento angolare di un corpo rigido. 
%Momento di inerzia e calcolo di momenti di inerzia. 
Sistemi di forze equivalenti. 
%Relazioni sui momenti angolari, rispetto a poli e a sistemi di riferimento diversi. 
%Generalizzazione del teorema delle forze vive. 
%Problemi unidimensionali: corpo rigido in rotazione attorno ad un asse fisso; pendolo fisico; rotolamenti; carrucole, rocchetti. 



Fenomeni d'urto

%Interazione di breve durata: urti e salti; forze impulsive e approssimazione impulsiva. 
Urti tra sistemi rigidi. 
%Annullarsi della variazione della quantità di moto totale e del momento angolare totale. 
%Urti elastici. 
%Urti anelastici.


\subsection{Conseguenze dei Principi della Dinamica}

\paragraph{Quantità di Moto e Impulso}
\paragraph{Momento Angolare e Momento della Forza}
\paragraph{Lavoro e Potenza}
\paragraph{Teorema dell'Energia Cinetica}
\subsubsection{Campi di Forze}
\paragraph{Energia Potenziale e Meccanica}


\subsection{Gravitazione}
\paragraph{Massa Gravitazionale}
\paragraph{Campo Gravitazionale}
\paragraph{Leggi di Keplero}



\section{Dinamica dei Sistemi}
\subsection{Conseguenze dei Principi della Dinamica sui Sistemi}
\paragraph{Centro di Massa}
\subsection{Teoremi sui Sistemi}
\paragraph{Equazioni Cardinali}
\paragraph{Teoremi di Konig}
\paragraph{Generalizzazione del Teorema delle Forze Vive}
\paragraph{Sistemi Continui}




\section{Meccanica dei Corpi Rigidi}

\subsection{Cinematica dei Corpi Rigidi}


\subsection{Equazioni Cardinali dei Corpi Rigidi}


\subsection{Teoremi di Konig per i Corpi Rigidi}

\paragraph{Energia Cinetica}
\paragraph{Momento Angolare e Momento delle Forze}