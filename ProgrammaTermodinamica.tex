


Termologia

%Fenomenologia dei fenomeni termici e definizione empirica delle quantità termodinamiche.
%Concetto di pressione Termometrica. 
%Sistemi termodinamici. 
%Equilibrio termodinamico. 
%Pareti conduttrici e adiabatiche. 
%Equilibrio termico e Principio zero della termodinamica. 
%Definizione di temperatura. 
%Vari tipi di scale termometriche. 
%Variabili intensive ed estensive.



Calorimetria

%Calore e quantità di calore. 
%Calori specifici. 
%Calori latenti nei cambiamenti di stato. 
%Misura delle quantità di calore e calorimetri.



Trasformazioni termodinamiche

%Trasformazioni termodinamiche. 
%Trasformazioni termodinamiche quasi statiche e brusche, reversibili e irreversibili.
%Piano di Clapeyron e rappresentazione grafica degli stati di equilibrio e delle trasformazioni quasi statiche e brusche. 
%Esempi di trasformazioni. 
%Lavoro in una trasformazione termodinamica.

Il primo principio della termodinamica

%L'equivalente meccanico della caloria: esperienza di Joule. 
%Equivalenza tra calore e lavoro. 
%Il primo principio della termodinamica. 
%Energia interna.



Gas perfetti (e gas reali)

%Esperienze di Gay-Lussac e Boyle-Mariotte. 
%Coefficienti di espansione e di comprimibiltà dei gas. 
%Equazione di stato dei gas perfetti. Trasformazioni termodinamiche  gas perfetti. 
%Calori specifici (molari) dei gas perfetti. 
%Esperienza di Joule: energia interna di un gas perfetto. 
%Relazione di Mayer. 
%Trasformazioni adiabatiche di un gas perfetto. 
%Il ciclo di Carnot con un gas perfetto. 
%I gas reali: isoterme dei gas reali. 
Fenomenologia delle trasformazioni di fase.



Il secondo principio della termodinamica

%Enunciati di Clausius e di Kelvin del secondo principio della termodinamica. 
%Equivalenza degli enunciati. 
%Macchine termiche e frigorifere. 
%Cicli termodinamici. 
%Teorema di Carnot e temperatura termodinamica assoluta. 
%Integrale di Clausius e teorema di Clausius. 
%Somma e integrale di Clausius. 
%Definizione della funzione di stato entropia. 
%Entropia e secondo principio della termodinamica. 
%Entropia crescente dell'universo termodinamico. 
%Calcoli di variazioni di entropia in varie trasformazioni dei gas ideali.  