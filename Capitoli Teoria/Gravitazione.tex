\documentclass{article}
\usepackage[utf8]{inputenc}
\usepackage{amsmath}
\usepackage{geometry}
\geometry{
 a4paper,
 total={182mm,257mm},
 left=14mm,
 top=20mm,
 }
 \usepackage[utf8]{inputenc}
 \usepackage[italian]{babel}
\usepackage[T1]{fontenc}
\usepackage{amssymb}
\usepackage{physics}
\usepackage{tikz} 
\usepackage{graphicx}
\graphicspath{ {Immagini/} }
\usepackage{float}
\usepackage{hyperref}
\hypersetup{
    colorlinks=true,
    linkcolor=red,
    citecolor=green
    filecolor=magenta,      
    urlcolor=cyan,
}


%Theorem Environments
\newtheorem{thm}{Teorema}[section]
\newtheorem{lem}[thm]{Lemma}
\newtheorem{property}{Proprietà}[section]
\newtheorem{defn}{Definizione}[section]
\newtheorem{prop}[defn]{Proposizione}
\newtheorem{example}{Esempi}[subsection]
\newtheorem{exerc}[example]{Esercizi Svolti}

%Commandi di Formattazione
\newcommand{\noi}{\noindent}
\newcommand{\note}{\noindent {\quad \bf \underline{Osservazione:}} \quad}
\newcommand{\eg}{\noindent {\bf \underline{Esempio:}} \quad}
\newcommand{\bfemph}[1]{\textbf{\textit{#1}}}
\renewcommand{\emph}[1]{\bfemph{#1}}

%Number Sets
\newcommand{\R}{\mathbb{R}}
\newcommand{\C}{\mathbb{C}}
\newcommand{\Z}{\mathbb{Z}}
\newcommand{\Q}{\mathbb{Q}}

%Shortcuts
\newcommand{\then}{\ensuremath{\Rightarrow}}
\newcommand{\twopartdef}[4]
{
	\left\{
		\begin{array}{ll}
			#1 & \mbox{se } #2 \\
			#3 & \mbox{se } #4
		\end{array}
	\right.
}

%Vectors
\renewcommand{\i}{\vu{i}}
\renewcommand{\j}{\vu{j}}
\renewcommand{\k}{\vu{k}}
\renewcommand{\a}{\va{a}}
\renewcommand{\b}{\va{b}}
\renewcommand{\c}{\va{c}}
\renewcommand{\v}{\va{v}}
\renewcommand{\u}{\va{u}}
\newcommand{\s}{\va{s}}
\renewcommand{\t}{\va{t}}
\newcommand{\verst}{\vu{t}}
\newcommand{\versr}{\vu{r}}
\renewcommand{\r}{\va{r}}
\newcommand{\tauvs}{\vu{\tau}}
\newcommand{\tauvt}{\va{\tau}}
\newcommand{\normvs}{\vu{n}}
\newcommand{\N}{\va{N}}
\newcommand{\g}{\va{g}}
\newcommand{\F}{\va{F}}
\newcommand{\f}{\va{f}}
\newcommand{\M}{\va{M}}
\renewcommand{\l}{\va{l}}
\newcommand{\p}{\va{p}}
\renewcommand{\P}{\va{P}}
\renewcommand{\L}{\va{L}}

\title{Appunti Lezioni}
\author{Roberto Gargiulo}
\date{Ultimo Aggiornamento: \today}


\begin{document}

\maketitle
\tableofcontents
\pagebreak

\section{Lezioni di Meccanica II Semestre}











\subsection{Campi Centrali e Momento Angolare - 6 Aprile}

Si definisce \textbf{Campo Centrale} un Campo di Forze per cui, dato un centro O, la direzione della forza applicata in un qualunque punto dello spazio è sempre la congiungente tra quel punto e il centro O, tali forze possono essere sia repulsive che attrattive e si possono scrivere con la seguente formulazione:
\begin{equation}
    \boxed{\F(\r)=f(|\r|)\r}
\end{equation}
Un esempio tipico è la forza gravitazionale:
\[\F_g(\r)=G\frac{mM}{|\r|^3}\r\then f(|\r|)=G\frac{mM}{|\r|^3}\]
Il \textbf{momento di un vettore} rispetto ad un punto O si definisce come segue:
\begin{equation}
    \boxed{\M=\r\times\a}
\end{equation}
Dove $\r$ è il vettore posizione del punto in cui è applicato il vettore $\a$ rispetto ad O.
\begin{defn}[Momento Angolare]
Il momento angolare è il momento della quantità di moto:
\[\l_O=\r\times\p=\r\times(m\v)\]
\end{defn}
\begin{defn}[Momento di una Forza]
\[\M_O=\r\times\F\]
\end{defn}
\begin{property}[Variazione del Momento Angolare]
\begin{equation}
    \dv{\l_O}{t}=m\dv{\r\times\v}{t}=m\dv{\r}{t}\times\v+m\r\times\dv{\v}{t}=m(\v\times\v+\r\times\a)=\r\times(m\a)=\r\times\F=\M_O
\end{equation}
Ossia la variazione del momento angolare è determinata dall'azione di una forza esterna sul sistema.
\end{property}
\begin{thm}[Conservazione del Momento Angolare]
Se la forza agente sul sistema è una forza centrale allora otteniamo:
\[\dv{\l_O}{t}=\r\times\F(\r)=\r\times(f(|\r|)\r=f(|\r|)\r\times\r=0\iff\boxed{\l_O=costante}\]
Ossia se la forza agente sul corpo è centrale rispetto al punto in cui si calcola il momento angolare tale momento angolare rimane \textbf{costante}.
\end{thm}
Ciò spiega numerosi fenomeni (ex.: la conservazione del momento angolare di un pianeta).
Inoltre spiega anche perchè i corpi (e in particolari i pianeti) tendono a ruotare in un piano, in quanto il momento angolare rimane perpendicolare ai vettori $\r,\v$ (per definizione di momento angolare) e non cambia la propria direzione, verso e modulo, quindi i vettori $\r,\v$ non possono cambiare piano.

\paragraph{La Seconda Legge di Keplero}
La conservazione del momento angolare dal punto di vista del modulo garantisce invece che la \textbf{velocità areolare} è la stessa in ogni punto della traiettoria, indipendentemente dalla sua forma. Consideriamo l'area infinitesima determinata dal prodotto vettoriale tra il raggio vettore e lo spostamento infinitesimo $\dd\r$:
\begin{equation}
|\r\times\dd\r|=\dd A\then \dv{A}{t}=\frac{1}{2}\left|\r\times\dv{\r}{t}\right|=\frac{|\l_O|}{2m}=costante=v_A
\end{equation}
Ossia la velocità areolare è costante.
\subsection{Gravitazione e Introduzione alla Dinamica dei Sistemi - 9 Aprile}
\paragraph{La III Legge di Keplero}
Notiamo che per ogni traiettoria circolare di un corpo soggetto alla forza gravitazionale (centrale) il moto è necessariamente uniforme:
\[r=costante\quad \frac{|\l_O|}{m}=|\r\times\v|=rv=\then v=costante\]
%Possiamo calcolare quindi l'energia meccanica uguale in ogni istante.
Si può ottenere quindi la terza legge notando che l'unica forza agente sul sistema è la forza gravitazionale che risulta uguale alla forza centripeta in quanto il corpo si muove di moto circolare uniforme:
\[G\frac{mM}{d^2}=md\omega^2=md\frac{4\pi^2}{T^2}\then \frac{d^3}{T^2}=\frac{GM}{4\pi^2}=k \]
Ossia il cubo della distanza è proporzionale al quadrato del periodo.
\paragraph{Velocità di Fuga}
Consideriamo un corpo che si trova sulla superficie terrestre, vogliamo fornirgli l'energia necessaria per tale da fermarsi a distanza infinita, ossia tale che la sua energia meccanica si annulli:
\[E=\frac{1}{2}m\cdot0-0=0\]
Siccome l'energia meccanica si conserva in un campo centrale (in quanto anche conservativo) allora possiamo determinare la velocità di fuga come segue:
\begin{equation}
    E=\frac{1}{2}mv^2-G\frac{mM}{R_T}=0\then \boxed{v=\sqrt{\frac{2GM}{R_T}}}
\end{equation}
E introducendo l'accelerazione di gravità $\g$ con modulo $g=\frac{GM}{R_T^2}$ otteniamo:
\begin{equation}
    \boxed{v=\sqrt{2gR_T}}
\end{equation}
\paragraph{Afelio e Perielio}
Per conoscere velocità e posizione dell'Afelio e Perielio di un'orbita è necessario imporre tre condizioni:
\[\left\{\begin{array}{l}
    v_Pr_P=v_Ar_A  \\
    E_P=E_A  \\
    r_P-r_A=a  
\end{array}\right.\]
Ossia la conservazione del momento angolare (in modulo), la conservazione dell'energia meccanica e la differenza tra la distanza al perielio e all'afelio come la distanza focale.











\end{document}