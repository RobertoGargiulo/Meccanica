\documentclass{article}
\usepackage[utf8]{inputenc}
\usepackage{amsmath}
\usepackage{geometry}
\geometry{
 a4paper,
 total={182mm,257mm},
 left=14mm,
 top=20mm,
 }
 \usepackage{amsthm}
 \usepackage[utf8]{inputenc}
 \usepackage[italian]{babel}
\usepackage[T1]{fontenc}
\usepackage{amssymb}
\usepackage{physics}
\usepackage{commath}
\usepackage{tikz}
\usepackage{pgfplots}
\usepackage{graphicx}
\graphicspath{ {Immagini/} }
\usepackage{float}
\usepackage{hyperref}
\hypersetup{
    colorlinks=true,
    linkcolor=red,
    citecolor=green
    filecolor=magenta,      
    urlcolor=cyan,
}


%Theorem Environments
\newtheorem{thm}{Teorema}[section]
\newtheorem{lem}[thm]{Lemma}
\newtheorem{property}{Proprietà}[section]
\newtheorem{defn}{Definizione}[section]
\newtheorem{prop}[defn]{Proposizione}
\newtheorem{example}{Esempi}[subsection]
\newtheorem{exerc}[example]{Esercizi Svolti}

%Commandi di Formattazione
\newcommand{\noi}{\noindent}
\newcommand{\note}{\noindent {\quad \bf \underline{Osservazione:}} \quad}
\newcommand{\eg}{\noindent {\bf \underline{Esempio:}} \quad}
\newcommand{\bfemph}[1]{\textbf{\textit{#1}}}
\renewcommand{\emph}[1]{\bfemph{#1}}

%Number Sets
\newcommand{\R}{\mathbb{R}}
\newcommand{\C}{\mathbb{C}}
\newcommand{\Z}{\mathbb{Z}}
\newcommand{\Q}{\mathbb{Q}}

%Shortcuts
\newcommand{\then}{\ensuremath{\Rightarrow}}
\newcommand{\twopartdef}[4]
{
	\left\{
		\begin{array}{ll}
			#1 & \mbox{se } #2 \\
			#3 & \mbox{se } #4
		\end{array}
	\right.
}

%Vectors
\renewcommand{\i}{\vu{i}}
\renewcommand{\j}{\vu{j}}
\renewcommand{\k}{\vu{k}}
\renewcommand{\a}{\va{a}}
\renewcommand{\b}{\va{b}}
\renewcommand{\c}{\va{c}}
\renewcommand{\v}{\va{v}}
\renewcommand{\u}{\va{u}}
\newcommand{\s}{\va{s}}
\renewcommand{\t}{\va{t}}
\newcommand{\verst}{\vu{t}}
\newcommand{\versr}{\vu{r}}
\renewcommand{\r}{\va{r}}
\newcommand{\tauvs}{\vu{\tau}}
\newcommand{\tauvt}{\va{\tau}}
\newcommand{\normvs}{\vu{n}}
\newcommand{\N}{\va{N}}
\newcommand{\g}{\va{g}}
\newcommand{\F}{\va{F}}
\newcommand{\f}{\va{f}}
\newcommand{\M}{\va{M}}
\renewcommand{\l}{\va{l}}
\newcommand{\p}{\va{p}}
\renewcommand{\P}{\va{P}}
\renewcommand{\L}{\va{L}}


\renewcommand{\c}{\overline{c}}
\title{Introduzione - Termodinamica}
\author{Roberto Gargiulo}
\date{Ultimo Aggiornamento: \today}


\begin{document}

\maketitle
\tableofcontents


\section{Termodinamica}
\subsection{Termologia}
Lo studio della termodinamica si basa sulla possibilità di raggiungere un'equilibrio, enunciata dal \textbf{principio zero}. Gli equilibri termodinamici compongono uno \textbf{spazio termodinamico} descritto da un certo numero di variabili (dette \textbf{grandezze termodinamiche}), tra le quali:
\begin{enumerate}
    \item Pressione
    \item Temperatura
    \item Volume
    \item Energia Interna
    \item Entropia
\end{enumerate}
\begin{defn}[Equilibrio Termodinamico]
L'equilibrio di un sistema termodinamico è costituito da tre equilibri che il sistema deve raggiungere:
\begin{enumerate}
    \item \textbf{Equilibrio Meccanico}, a cui corrisponde l'assenza di movimento macroscopico ed è caratterizzato dalla pressione, si raggiunge quando la pressione esterna è uguale a quella interna.
    \item \textbf{Equilibrio Chimico}, a cui corrisponde l'assenza di reazioni chimiche e trasformazioni di fase, a cui corrisponde il potenziale chimico, si raggiunge quando il potenziale chimico dei reagienti $\mu_A$ è uguale a quello dei prodotti $\mu_B$.
    \item \textbf{Equilibrio Termico}, quando la temperatura dell'ambiente e quella del sistema sono le stesse.
\end{enumerate}
\end{defn}
Queste tre grandezze sono \textbf{intensive} in quanto non dipendono dalle dimensioni del sistema. Le grandezze che dipendono dalle dimensioni sono dette \textbf{estensive} (additive) e includono massa, volume, entropia ed energia interna.
L'equazione di stato di un sistema termodinamico è un'equazione del tipo:
\[t=t(p,V)\quad p=p(t,V)\]
Ossia una relazione tra pressione, volume e temperatura.
\begin{defn}[Varianza]
La varianza di un sistema termodinamico corrisponde al numero di gradi di libertà di quel sistema, ossia il numero di grandezze necessario a descriverlo. 
\end{defn}
Utilizzando la \textbf{Regola di Gibbs} possiamo calcolare la varianza di un sistema come segue:
\begin{equation}
    \boxed{v=n-f+2}
\end{equation}
Dove n è il numero di componenti (sostanze) ed f il numero di fasi. Quando $v=0$ il sistema si trova nel suo \textbf{punto triplo}, ossia un punto particolare dello spazio termodinamico determinato da specifiche coordinate di volume, pressione e temperatura. 

\subsection{Calorimetria e Principio Zero}

Consideriamo la quantità di calore infinitesima fornita ad un corpo che non sta subendo cambiamenti di fase, possiamo definire la grandezza \textbf{capacità termica} come:
\[C=\dv{Q}{t}\quad\dif t\neq 0\]
Si osserva empiricamente che tale grandezza è proporzionale alla massa e quindi al numero di moli secondo una costante detto \textbf{calore specifico} (di una sostanza) o calore specifico molare rispettivamente:
\[C=mc=n\overline{c}\]
Inoltre possiamo definire il \textbf{calore latente} per un sistema che sta subendo un cambiamento di fase:
\[\lambda=\dv{Q}{m}\]\\



\end{document}