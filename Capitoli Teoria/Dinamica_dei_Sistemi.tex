\documentclass{article}
\usepackage[utf8]{inputenc}
\usepackage{amsmath}
\usepackage{geometry}
\geometry{
 a4paper,
 total={182mm,257mm},
 left=14mm,
 top=20mm,
 }
 \usepackage[utf8]{inputenc}
 \usepackage[italian]{babel}
\usepackage[T1]{fontenc}
\usepackage{amssymb}
\usepackage{physics}
\usepackage{tikz} 
\usepackage{graphicx}
\graphicspath{ {Immagini/} }
\usepackage{float}
\usepackage{hyperref}
\hypersetup{
    colorlinks=true,
    linkcolor=red,
    citecolor=green
    filecolor=magenta,      
    urlcolor=cyan,
}


%Theorem Environments
\newtheorem{thm}{Teorema}[section]
\newtheorem{lem}[thm]{Lemma}
\newtheorem{property}{Proprietà}[section]
\newtheorem{defn}{Definizione}[section]
\newtheorem{prop}[defn]{Proposizione}
\newtheorem{example}{Esempi}[subsection]
\newtheorem{exerc}[example]{Esercizi Svolti}

%Commandi di Formattazione
\newcommand{\noi}{\noindent}
\newcommand{\note}{\noindent {\quad \bf \underline{Osservazione:}} \quad}
\newcommand{\eg}{\noindent {\bf \underline{Esempio:}} \quad}
\newcommand{\bfemph}[1]{\textbf{\textit{#1}}}
\renewcommand{\emph}[1]{\bfemph{#1}}

%Number Sets
\newcommand{\R}{\mathbb{R}}
\newcommand{\C}{\mathbb{C}}
\newcommand{\Z}{\mathbb{Z}}
\newcommand{\Q}{\mathbb{Q}}

%Shortcuts
\newcommand{\then}{\ensuremath{\Rightarrow}}
\newcommand{\twopartdef}[4]
{
	\left\{
		\begin{array}{ll}
			#1 & \mbox{se } #2 \\
			#3 & \mbox{se } #4
		\end{array}
	\right.
}

%Vectors
\renewcommand{\i}{\vu{i}}
\renewcommand{\j}{\vu{j}}
\renewcommand{\k}{\vu{k}}
\renewcommand{\a}{\va{a}}
\renewcommand{\b}{\va{b}}
\renewcommand{\c}{\va{c}}
\renewcommand{\v}{\va{v}}
\renewcommand{\u}{\va{u}}
\newcommand{\s}{\va{s}}
\renewcommand{\t}{\va{t}}
\newcommand{\verst}{\vu{t}}
\newcommand{\versr}{\vu{r}}
\renewcommand{\r}{\va{r}}
\newcommand{\tauvs}{\vu{\tau}}
\newcommand{\tauvt}{\va{\tau}}
\newcommand{\normvs}{\vu{n}}
\newcommand{\N}{\va{N}}
\newcommand{\g}{\va{g}}
\newcommand{\F}{\va{F}}
\newcommand{\f}{\va{f}}
\newcommand{\M}{\va{M}}
\renewcommand{\l}{\va{l}}
\newcommand{\p}{\va{p}}
\renewcommand{\P}{\va{P}}
\renewcommand{\L}{\va{L}}

\title{Dinamica dei Sistemi di Corpi}
\author{Roberto Gargiulo}
\date{Ultimo Aggiornamento: \today}


\begin{document}

\maketitle

\section{Introduzione ai Sistemi di Corpi Discreto}

Un \textbf{sistema di corpi} si definisce come segue:
\begin{defn}[Sistema di Corpi a Massa Concentrata]
una porzione dello spazio che include un certo numero n di corpi.
\end{defn}
Notiamo che sia le forze (interazioni esercitata tra i corpi interni al sistema) che i momenti interni si oppongono e sono diretti lungo la retta congiungente:
\[\f_{ij}=-\f{ji}\]
\[\M_{Oi}=\r_i\times\f_{ij}=-\r_j\times\f_{ji}=-\M_{Oj}\]

L'equazione del moto del corpo i-esimo è data quindi da:
\begin{equation}
    m_i\a_i=\sum_A\F_{iA}+\sum_j\f_{ij}=\F_i^{(e)}+\F_i^{(i)}
\end{equation}
Quindi il sistema è definito da n equazioni vettoriali (ossia 3n equazioni scalari).

\paragraph{Proprietà Globali di un Sistema}
Definiamo quindi alcune grandezze utili per analizzare il comportamento di un sistema di corpi:
\begin{defn}[Quantità di Moto Totale]
\[\P=\sum_im_i\v_i=\sum_i\p_i\]
\end{defn}
\begin{defn}[Momento Angolare Totale]
\[\L_O=\sum_i\r_i\times\p_i=\sum_i\l_{Oi}\]
\end{defn}
\begin{defn}[Energia Cinetica Totale]
\[K=\sum_i\frac{1}{2}m_iv_i^2\]
\end{defn}
\begin{defn}[Massa Totale del Sistema]
\[M=\sum_im_i\]
\end{defn}
Notiamo che possiamo dividere la quantità di moto totale per la massa totale, ottenendo una "media pesata" delle velocità dei corpi:
\[\v_C=\frac{\P}{M}=\frac{\sum_im_i\v_i}{\sum_im_i}=\sum_i\frac{m_i}{M}\v_i=\sum_i\pi_i\v_i\quad\sum_i\pi_i=1\]
Segue che:
\[\v_C=\sum_i\pi_i\dv{\r_i}{t}=\dv{\sum_i\pi_i\r_i}{t}\dv{\r_C}{t}\]
Ha senso quindi definire il centro di massa come segue:
\begin{defn}[Centro di Massa]
\[\r_C=\sum_i\frac{m_i}{M}\r_i\]
\end{defn}
\section{I e II Equazione Cardinale}
\begin{property}[I Equazione Cardinale]
Consideriamo un sistema di corpi, allora la derivata della quantità di moto totale vale:
\begin{equation}
\begin{split}
    \dv{\P}{t}=M\dv{\v_C}{t}&=M\a_C=\dv{\sum_im_i\v_i}{t}=\sum_im_i\dv{\v_i}{t}=\sum_im_i\a_i=\sum_i\left(\F_i^{(e)}+\F_i^{(i)}\right)=\sum_iF_i^{(e)}=\F^{(e)}\then \boxed{\dv{\P}{t}=\F^{(e)}}
\end{split}    
\end{equation}
Dove:
\[\sum_i\F_i^{(i)}=\sum_{i,j=1}^n\f_{ij}=-\sum_{i,j}\f_{ji}=-\sum_{i,j}\f_{ij}\iff \sum_{i,j}\f_{ij}=0\]
\end{property}
\note La I Equazione Cardinale afferma che il moto del centro di massa è determinato esclusivamente dalle forze esterne, in quanto quelle interne non hanno effetto. 
Un sistema in cui la quantità di moto totale si conserva ha particolare importanza, definiamo come segue un tale sistema di corpi:
\begin{defn}[Sistema Isolato]
Un \textbf{sistema isolato} è un sistema tale che $\F^{(e)}=0$, ossia la quantità di moto totale si conserva e quindi il centro di massa si muove di moto rettilineo uniforme.
\end{defn}
\begin{property}[II Equazione Cardinale]
Consideriamo un sistema di corpi discreto, allora la derivata del momento angolare totale vale:
\begin{equation}
\begin{split}
\dv{\L_O}{t}&=\dv{}{t}\left[\sum_i\r_i\times(m_i\v_i)\right]=\\
&=\sum_i\left[\dv{\r_i\times(m_i\v_i)}{t}\right]=\sum_i\left[\dv{(\r_i}{t}\times(m_i\v_i))+\r_i\times m_i\dv{\v_i}{t}\right]=\\
&=\sum_i\left[\v_i\times(m_i\v_i)+\r_i\times m_i\a_i\right]=\sum_i\left[\r_i\times(m_i\a_i)\right]=\\
&=\sum_i\left[\r_i\times\left(\F_i^{(e)}+\sum_j\f_{ij}\right)\right]=\sum_i\left(\r_i\times\F_i^{(e)}\right)+\sum_{i}\left(\r_i\times\sum_j\f_{ij}\right)=\\
&=\sum_i\left(\r_i\times\F_i^{(e)}\right)=\M_O^{(e)}\then \boxed{\dv{\L_O}{t}=\M_O^{(e)}}
\end{split}
\end{equation}
Dove:
\begin{equation}
\begin{split}
    \sum_i\left(\r_i\times\sum_j\f_{ij}\right)&=\sum_{i,j}\r_i\times\f_{ij}=\sum_{i,j}\r_j\times\f_{ji}=\sum_{i,j}\r_j\times(-\f_{ij})\then\\ 
    \then 2\sum_i\left(\r_i\times\sum_j\f_{ij}\right)=&\sum_{i,j}(\r_i-\r_j)\times\f_{ij}=0\quad\quad (\r_i-\r_j)\parallel\f_{ij}
\end{split}
\end{equation}
\end{property}
\begin{property}[La Variazione di Energia Cinetica]
La variazione di energia cinetica del centro di massa vale:
\[\Delta K=L^{(e)}+L^{(int)}\]
Il lavoro infinitesimo vale invece:
\begin{equation}
    \dd L^{(e)}=\sum_i\left(\F_i^{(e)}\dd\r_i\right)\quad\dd L^{(int)}=\sum_{i,j}\f_{ij}\dd\r_i\then
\end{equation}
\begin{equation}
\begin{split}
    \then\dd L&=\dd L^{(e)}+\dd L^{(int)}=\sum_i\left[\left(\F_i^{(e)}+\sum_j\f_{ij}\right)\dd\r_i\right]=\\
    &=\sum_i\F_i\dd\r_i=\sum_i\dd K_i=\dd\sum_iK_i=\dd K\iff L=\Delta K
\end{split}
\end{equation}
Calcoliamo quanto vale il lavoro delle forze interne sul corpo i-esimo:
\begin{equation}
\begin{split}
    \dd L^{(int)}&=\sum_{i,j}f_{ij}\dd\r_i=\sum_{i,j}\f_{ji}\dd\r_j=\\
    &=-\sum_{i,j}f_{ij}\dd\r_j\then 
\end{split}
\end{equation}
\begin{equation}
\begin{split}
    \then 2\dd L^{(int)}=&\sum_{i,j}\f_{ij}(\dd\r_i-\dd\r_j)=\sum_{i,j}\f_{ij}\dd(\r_i-\r_j)=\\
    &=\sum_{i,j}\sigma_{ij}(\r_i-\r_j)\dd(\r_i-\r_j)=\sum_{i,j}\frac{1}{2}\sigma_{ij}\dd|\r_i-\r_j|^2=\\
    =\frac{1}{2}\sum_{i,j}\sigma_{ij}\dd D_{ij}
\end{split}    
\end{equation}
\note Possiamo esprimere l'interazione $\f_{ij}$ tra due corpi interni al sistema come il prodotto tra una costante $\sigma_{ij}$ e il vettore congiungente i due corpi (ossia $\r_i-\r_j$). Inoltre poniamo $D_{ij}=|\r_i-\r_j|^2$. Se questa quantità è costante allora la distanza tra i corpi non cambia, questo tipo di sistema è detto \textbf{corpo rigido}.
\end{property}
\paragraph{Sistema a due Corpi}
Il sistema a due corpi si può risolvere riducendo il sistema ad un solo corpo, e utilizzando la \textbf{massa ridotta del sistema}:
\begin{equation}
    \frac{1}{\mu}=\frac{1}{m_1}+\frac{1}{m_2}\then\boxed{\mu=\frac{m_1m_2}{m_1+m_2}}
\end{equation}
Impostiamo le equazioni del moto per i due corpi:
\begin{equation}
\left\{\begin{array}{l}
    m_1\a_1=\F_1^{(e)}+\f_{12}=\F_1^{(e)}+\f  \\
    m_2\a_2=\F_2^{(e)}+\f_{21}=\F_2^{(e)}-\f 
\end{array}\right.\then m_1\a_1+m_2\a_2=\F_1^{(e)}+\F_2^{(e)}=\F^{(e)}=M\a_C=(m_1+m_2)\a_C\then
\end{equation}
\begin{equation}
    m_1\a_1+m_2\a_2=(m_1+m_2)\a_C\quad \a_1=\frac{\F_1^{(e)}}{m_1}+\frac{\f}{m_1}\quad \a_2=\frac{\F_2}{m_2}-\frac{\f}{m_2}
\end{equation}
Otteniamo quindi:
\begin{equation}
    \dv[2]{(\r_1-\r_2)}{t}=\dv[2]{\r_{12}}{t}=\a_{12}=\a_1-\a_2=\frac{\F_1}{m_1}-\frac{\F_2}{m_2}+\left(\frac{1}{m_2}+\frac{1}{m_2}\right)\then \mu\a_{12}=\frac{m_2}{m_1+m_2}\F_1-\frac{m_1}{m_1+m_2}\F_2+\f 
\end{equation}
Dove: \[\f=\f(\r_1-\r_2)\]
Ossia la forza con cui interagiscono i due corpi dipende dalla distanza relativa tra i corpi, non quella assoluta. 
\paragraph{Sistema Isolato a Due Corpi}
Quando il sistema è isolato otteniamo le seguenti equazioni differenziali:
\begin{equation}
\left\{\begin{array}{l}
    \mu\ddot{\r}_{12}=\f(\r_{12}) \\
    M\ddot{\r}_C=0
\end{array}
\right.
\end{equation}
Da cui le soluzioni:
\begin{equation}
\left\{\begin{array}{l}
    \r_{12}(t)=\r_1(t)-\r_2(t)  \\
     \r_C(t)=\r_{C0}+\v_{C0}t=\frac{m_1\r_1+m_2\r_2}{m_1+m_2} 
\end{array}\right.
\end{equation}
Da cui possiamo ricavare l'espressione di $\r_1,\r_2$ in funzione di t.
\paragraph{Sistema del Centro di Massa}
Il Sistema del centro di Massa è il sistema di riferimento più comodo per la maggior parte dei casi. È un sistema centrato nel centro di massa e tale che il centro di massa sia istantaneamente fermo ma tale che il sistema si muova insieme al centro e sia comunque inerziale, fissando l'orientamento rispetto alle stelle fisse. Altre caratteristiche rilevanti del sistema del centro di massa sono:
\begin{align*}
    \sum_im_i\r_i=0\\
    \sum_im_i\v_i=\va{P'}=0
\end{align*}
Infatti:
\begin{equation}
\begin{split}
    \sum_im_i\v_i'&=\sum_im_i(\v_i-\v_c)=\sum_im_i\v_i-\sum_im_i\frac{\sum_jm_j\v_j}{\sum_im_i}=\\
    &=\sum_im_i\v_i-\sum_jm_j\v_j=\va{P}-\va{P}=0
\end{split}
\end{equation}

\section{I e II Teorema di Konig - Dinamica nei Sistemi nei Sistemi di Riferimento}
\paragraph{II Teorema di König}
Consideriamo un sistema inerziale K e un sistema in moto rispetto a K detto K', allora la velocità rispetto a K del corpo i-esimo di un sistema di corpi si può calcolare come:
\[\v_i=\v_i'+\va{V}\]
Dove $\va{V}$ è la velocità di K' rispetto a K. Pertanto possiamo esplicitare l'energia cinetica del sistema rispetto a K come:
\begin{equation}
\begin{split}
    K&=\frac{1}{2}\sum_i\v_i^2=\frac{1}{2}\sum_im_i(\v_i'+\va{V})^2=\\
    &=\frac{1}{2}\sum_im_iv_i'^2+\frac{1}{2}\sum_im_iV^2+\frac{1}{2}\sum_im_i\v_i'\va{V}=\\
    &=\sum_i\frac{1}{2}m_iv_i'^2+\left(\frac{1}{2}\sum_im_i\right)V^2+\left(\frac{1}{2}\sum_im_i\v_i'\right)\va{V}=\\
    &=\boxed{K'+\frac{1}{2}MV^2+\va{P'}\cdot\va{V}}
\end{split}    
\end{equation}
Scegliendo come sistema il sistema del centro di massa C otteniamo:
\begin{equation}
    \va{P'}=0\then \boxed{K=K'+\frac{1}{2}Mv_C^2}
\end{equation}
Che è proprio il II Teorema di König.


\paragraph{I Teorema di König}
Consideriamo ora il Momento Angolare rispetto a K utilizzando i valori di posizione e velocità in un sistema K' centrato in $\Omega$; otteniamo quindi:
\[\left\{\begin{array}{l}
    \r_i=\r_i'+\r_{\Omega}   \\
     \v_i=\v_i'+\v_{\Omega}
\end{array}\right.\]
Allora il momento angolare rispetto al centro di K vale:
\begin{equation}
\begin{split}
    \L_O&=\sum_i\r_i\times\v_i=\sum_im_i(\r_i'+\r_{\Omega})\times(\v_i'+\v_{\Omega})=\\
    &=\sum_im_i\r_i'\times\v_i'+\sum_im_i\r_i'\times\v_{\Omega}+\sum_im_i\r_{\Omega}\times\v_i'+\sum_im_i\r_{\Omega}\times\v_{\Omega}=\\
    &=\L_{\Omega}'+\left(\sum_im_i\r_i'\right)\times\v_{\Omega}+\r_{\Omega}\times\left(\sum_im_i\v_i'\right)+\r_{\Omega}\times(M\v_{\Omega})=\\
    &=\L_{\Omega}'+\left(\frac{\sum_im_i}{\sum_im_i}\sum_im_i\r_i\right)\times\v_{\Omega}+\r_\Omega\times\P'+\r_{\Omega}\times M\v_{\Omega}=\\
    &=\L_{\Omega}'+\left(M\frac{\sum_im_i\r_i}{\sum_im_i}\right)\times\v_{\Omega}+\r_\Omega\times\P'+\r_{\Omega}\times M\v_{\Omega}=\\
    &=\boxed{\L_{\Omega}'+M\r_C'\times\v_{\Omega}+\r_{\Omega}\times\P'+\r_{\Omega}\times M\v_{\Omega}}
\end{split}
\end{equation}
Se $\Omega=C$ allora otteniamo:
\[\v_{\Omega}=\v_C\quad\r_{\Omega}=\r_C\then \r_C'=0\quad\v_C'=0\]
Da cui:
\begin{equation}
    \L_O=\L_C'+\r_C'\times(\M\v_C)+\r_C\times\P'+\r_C\times(M\v_c)=\boxed{\L_C'+\r_C\times\P}
\end{equation}
Che è proprio il I Teorema di König.
\paragraph{Cambiamento di Polo}
La formula di cambiamento di polo per il momento angolare si ottiene considerando solo le posizioni rispetto al sistema mobile:
\begin{equation}
\begin{split}
    \L_O&=\sum_im_i(\r_i'+\r_{\Omega})\times\v_i=\sum_im_i\r_i'\times\v_i+\sum_im_i\r_{\Omega}\times\v_i=\\
    &=\boxed{\L_{\Omega}+\r_{\Omega}\times\P}
\end{split}
\end{equation}
Se $\Omega=C$ allora otteniamo anche:
\[\L_O=\L_C+\r_C\times\P=\L_C'+\r_C\times\P\]
Ossia $\L_C=\L_C'$, due grandezze fisicamente diverse ma numericamente uguali per natura del centro di massa.
\paragraph{Generalizzazione della II Equazione Cardinale}
Deriviamo la formula di cambiamento di polo per il momento angolare:
\begin{equation}
\begin{split}
    \M_O&=\dv{\L_O}{t}=\dv{\L_{\Omega}}{t}+\dv{\r_{\Omega}}{t}\times\P+\r_{\omega}\times\dv{\P}{t}=\\
    &=\dv{\L{\Omega}}{t}+\v_{\Omega}\times\P+\r_{\Omega}\times\F^{(e)}\then\\
    \then \dv{\L_{\Omega}}{t}&=\dv{\L_O}{t}-\r_{\Omega}\times\F^{(e)}-\v_{\Omega}\times\P=\\
    &=\M_O-\r_{\Omega}\times\F^{(e)}-\v_{\Omega}\times\P=\\
    &=\sum_i\left[\r_i\times\F_i^{(e)}-\r_{\Omega}\times\F_i^{(e)}\right]-\v_{\Omega}\times\P=\\
    &=\sum_i\left[(\r_i-\r_{\Omega})\times\F_i^{(e)}\right]-\v_{\Omega}\times\P=\\
    &=\sum_i\r_i'\times\F_i^{(e)}-\v_{\Omega}\times\P=\boxed{\M_{\Omega}^{(e)}-\v_{\Omega}\times\P}
\end{split}
\end{equation}
Scegliendo $\Omega=O$ otteniamo $\v_{\Omega}=0$ mentre scegliendo $\Omega=C$ otteniamo $\v_C\parallel\P$, ossia in entrambi i casi:
\[\dv{\L_O}{t}=\M_O^{(e)}\quad\dv{\L_C}{t}=\M_C^{(e)}\]


\paragraph{Esempio}
Se $\F_i^{(e)}=m_i\g$ allora otteniamo:
\[\dv{\P}{t}=\sum_i\F_i^{(e)}=\sum_im_i\g=M\g=M\a_c\iff \g=\a_C\]
Ossia il centro di massa si muove come un proiettile nel caso in cui l'unica forza esterna al sistema sia la forza-peso. Inooltre:
\[\M_O^{(e)}=\sum_i\r_i\times\F_i^{(e)}=\sum_i\r_i\times(m_i\g)=\sum_i(m_i\r_i)\times\g=M\r_c\times\g=\r_c\times(M\g)\]
\paragraph{Coppie di Forze}
Se $\F_1^{(e)}=-\F_2^{(e)}=\F$ allora il momento delle forze esterne agenti sul sistema è detto \textbf{momento di una coppia di forze}:
\begin{equation}
    \M_O^{(e)}=\r_1\times\F+\r_2\times(-\F)=(\r_1-\r_2)\times\F=\M^{(e)}
\end{equation}
Ossia il momento di una forza di coppie dipende solo dalla distanza relativa, detta \textbf{braccio}, e non dal polo/posizioni assolute.

\section{I Sistemi Continui}
I \textbf{sistemi continui} o a massa ridistribuita sono simili a quelli \textbf{discreti} (a massa concentrata) se non che al posto dei singoli corpi si considerano volumi infinitesimi $\dd V$ di massa $\dd M$ tali che:
\[\int\dd V=V\quad\quad\int\dd M=M\]
Da cui:
\[\p=\int\dd\p=\int\dd(m\v)=\int_M\dd(m\v(\r,t))\]
\[\L_O=\int\dd(m\r\times\v(\r,t))\]
Questi integrali sono detti \textbf{integrali di volume} che si possono ricondurre a quelli di linea definendo la \textbf{densità}:
\begin{defn}[Densità di Massa nel Punto $\r$ all'istante t]
\begin{equation}
    \rho(\r,t)=\dv{m}{V}
\end{equation}
\end{defn}
\note La funzione di densità (locale) dipende anche dal modo in cui ripartiamo il sistema. Generalmente si preferisce un cubetto infinitesimo che ha tre lati orientati lungo gli assi ottenendo:
\[\dd V=\dd x\dd y\dd z\]
Da cui:
\[\dd m=\rho(\r,t)\dd V=\rho(\r,t)\dd x\dd y\dd z\]
Possiamo quindi ridurre la dipendenza della quantità di moto totale al solo tempo:
\[\P=\int\v(\r,t)\dd m=\int\rho(\r,t)\v(\r,t)\dd V=\iiint_V\rho(x,y,z;t)\v(x,y,z;t)\dd x\dd y\dd z\]
Ossia rendiamo la quantità di moto un integrale triplo che dipende dalla frontiera del sistema, ossia la sua forma.\\
Possiamo inoltre definire la \textbf{densità media}:
\begin{equation}
    \overline{\rho}=\frac{M}{V}=\frac{\int_M\dd m}{\int_V\dd V}=\frac{\int_V\rho(\r,t)\dd V}{\int_V\dd V}
\end{equation}
Possiamo poi definire il \textbf{centro di massa} e la sua velocità:
\begin{equation}
    \r_C=\frac{1}{M}\int_M\r\dd m\quad\v_C=\frac{1}{M}\int\v(\r,t)\dd m
\end{equation}
Mentre in componenti:
\begin{align*}
    (\r_C)_x=\frac{1}{M}\int\rho(x,y,z)x\dd x\dd y\dd z\\
    (\r_C)_y=\frac{1}{M}\int\rho(x,y,z)z\dd x\dd y\dd z\\
    (\r_C)_z=\frac{1}{M}\int\rho(x,y,z)y\dd x\dd y\dd z\\
\end{align*}




\end{document}